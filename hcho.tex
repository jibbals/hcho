%% Copernicus Publications Manuscript Preparation Template for LaTeX Submissions
%% ---------------------------------
%% This template should be used for copernicus.cls
%% The class file and some style files are bundled in the Copernicus Latex Package, which can be downloaded from the different journal webpages.
%% For further assistance please contact Copernicus Publications at: production@copernicus.org
%% http://publications.copernicus.org/for_authors/manuscript_preparation.html
%% Please use the following documentclass and journal abbreviations for discussion papers and final revised papers.

\documentclass[acp, manuscript]{copernicus} % single column manuscript: used for both discussion and submission

\begin{document}
\title{Isoprene emissions over Australia}

% \Author[affil]{given_name}{surname}
\Author[1]{Jesse W.}{Greenslade}
\Author[1,2]{Jenny A.}{Fisher}

\affil[1]{Centre for Atmospheric Chemistry, School of Chemistry, University of Wollongong, Australia}
\affil[2]{School of Earth \& Environmental Sciences, University of Wollongong, Australia}

\runningtitle{Australian isorene emissions}
\runningauthor{Greenslade et al.}
\correspondence{Jesse Greenslade (jwg366@uowmail.edu.au)}

%% These dates will be inserted by Copernicus Publications during the typesetting process.
\firstpage{1}
\maketitle

\begin{abstract}
  
\end{abstract}

% Section 1 -- INTRO
\introduction  %% \introduction[modified heading if necessary]

  % AIMs paragraph


\section{Data and Model}
  \label{sec:DataModel}
    
    % EXAMPLE FIGURE
    %\begin{figure}[t]
    %  % Figure 4 
    %  % created in blah
    %  \includegraphics[width=8.3cm]{figures/filtereg.png}
    %  \caption{ %
    %	}
    %  \label{fig:process}
    %\end{figure}
   
    % EXAMPLE TABLE
%   \begin{table}[t]
%     %\centering
%     \caption{Total number of ozonesonde detected STT events, along with the number of events in each category (see text).}
%     \begin{tabular}{ c   c   c   c   c   c   c } 
%       \hline
%       Site & Events & Cut-offs & Frontals & Misc & Fire \\
%       \hline
%       % 41,   31,  28,  27
%       Davis       	& 80 & 44  & 19 & 17 & 0 \\ 
%       Macquarie Island 	& 105 & 19 & 31 & 34  & 21 \\
%       Melbourne 	& 127 & 28 & 31 & 41 & 27 \\
%       \hline
%     \end{tabular}
%     \label{table:EventCounts}
%   \end{table}
  
  
\conclusions  %% \conclusions[modified heading if necessary]


\appendix
% \section{}    %% Appendix A
% \subsection{}     %% Appendix A1, A2, etc.

\appendixfigures  %% needs to be added in front of appendix figures in one-column style (\documentclass[acp, manuscript]{copernicus})
\appendixtables   %% needs to be added in front of appendix tables in one-column style (\documentclass[acp, manuscript]{copernicus})

\authorcontribution{}

\competinginterests{The authors declare that they have no conflict of interest.}

%\disclaimer{disclaimer}

% Data availability
%
\textit{Data availability.} All GEOS-Chem model output is available from the authors upon request.

\begin{acknowledgements}
This research is supported by an Australian Government Research Training Program (RTP) Scholarship.
\end{acknowledgements}

%% Since the Copernicus LaTeX package includes the BibTeX style file copernicus.bst,
%% authors experienced with BibTeX only have to include the following two lines:
%%
\bibliographystyle{copernicus}
\bibliography{bibliography/hcho.bib}

\end{document}
